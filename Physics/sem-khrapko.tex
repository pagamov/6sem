%%% пример отчета взян с сайта http://k806.ystok.ru/

\documentclass[15pt]{extarticle}

\usepackage{fullpage}
\usepackage{multicol,multirow}
\usepackage{tabularx}
\usepackage{ulem}
\usepackage[utf8]{inputenc}
\usepackage[russian]{babel}
\usepackage{amsmath}
\usepackage{amssymb}

\usepackage[utf8]{inputenc}
\usepackage[russian]{babel}

%графики
\usepackage{pgfplots}
\pgfplotsset{compat=1.9}

% margin from paper borders
\usepackage[margin={0.8in, 0.3in}]{geometry}
% displaystyle в каждом мас моде
\everymath{\displaystyle}

\usepackage{titlesec}

\titleformat{\section}
  {\normalfont\Large\bfseries}{\thesection.}{0.3em}{}

\titleformat{\subsection}
  {\normalfont\large\bfseries}{\thesubsection.}{0.3em}{}

\titlespacing{\section}{0pt}{*2}{*2}
\titlespacing{\subsection}{0pt}{*1}{*1}
\titlespacing{\subsubsection}{0pt}{*0}{*0}
\usepackage{listings}
\lstloadlanguages{Lisp}
\lstset{extendedchars=false,
    breaklines=true,
    breakatwhitespace=true,
    keepspaces = true,
    tabsize=2
}

\begin{document}

\section*{Формулы по физике}

\section{Распределение}

\subsection{Среднее значение}

$<x> = \dfrac{\int{x * \varphi (x) dx}} {\int{\varphi (x) dx}}$

\subsection{Пример}

$\varphi (x) = x^2$

$0 \le x \le 1$

$<x> = \dfrac{\int{x * \varphi (x) dx}} {\int{\varphi (x) dx}} = \dfrac{\int{x^3 dx}} {\int{x^2 dx}} =  \left.\dfrac{1/4 * x^4} {1/3 * x^3} \right\vert_0^1 = 3/4$

\subsection{Распределение энергии теплового излучения по спектру, те по $\lambda$ или $\upsilon$}


$U \dfrac{\text{Дж}}{\text{м}^3}$ - объемная плотность.

$I \dfrac{\text{Дж}}{\text{м}^2 \text{сек}}$ - плотность потока.

$dU = U_{\lambda}(\lambda) d \lambda = U_{\upsilon}(\upsilon) d \upsilon$  - функция распределения

$dI = I_{\lambda}(\lambda) d \lambda = I_{\upsilon}(\upsilon) d \upsilon$  - функция распределения

Как связаны $U$ и $I$? Если бы $U$ неслась со скоростью света, то $I = U * c$, но тепловое излучение изотропно, во все стороны, то надо делить на два, и еще на два (?), откуда $I = \dfrac{U * c}{4} $

Какова связь распределений $U_{\lambda}(\lambda)$ и $U_{\upsilon}(\upsilon)$? 

$c = \lambda * \upsilon$, откуда для скалярной функции $U_{\lambda}(\lambda) = U_{\upsilon}(\dfrac{c}{\lambda})$

На самом деле $U_{\lambda}(\lambda) = U_{\upsilon}(\dfrac{c}{\lambda}) * \dfrac{d \upsilon}{d \lambda} = - U_{\upsilon}(\dfrac{c}{\lambda}) * \dfrac{c}{\lambda^2}$ так как

$\dfrac{d \upsilon}{d \lambda} = - \dfrac{c}{\lambda^2}$

Распределение Планка $U_{\lambda}(\lambda, t) = \dfrac{8 \pi h c}{\lambda^3 (e^{\dfrac{hc}{kt \lambda}} - 1)} \to_{\lambda \to 0} e^{-\dfrac{hc}{kt \lambda}} \to 0$


\begin{tikzpicture}
\begin{axis}[
	title = Plank,
	xlabel = {$x$},
	ylabel = {$y$},
	minor tick num = 2
]

\addplot [black] coordinates {
	(0,0) (-10,-20) (-14.311,-5.79) (5.39,-4.87)
};

\addplot [blue] coordinates {
	(0,0) (-40/3,-80/3) (-10.97,-4.95)
};

\addplot [green] coordinates {
	(0,0) (-4/10,-8/10) (-8/10,-33/25) (-293/250,-209/125) (-1881/1250,-12313/7500)
};

\addplot [red] coordinates {
	(-24/7 - 1/10,-20/7 + 1/10) (-24/7 + 1/10,-20/7 + 1/10) (-24/7 + 1/10,-20/7 - 1/10) (-24/7 - 1/10,-20/7 - 1/10) (-24/7 - 1/10,-20/7 + 1/10)
};

\end{axis}
\end{tikzpicture}

Крч центр купола в $\lambda^{*}(t)$

$\left. \dfrac{ d U_{\lambda}(\lambda)}{d \lambda} \right\vert_{\lambda^{*}} = 0$

$\lambda^{*} = \dfrac{b}{T}$ - Закон Вина

$b = 0.3 \text{см} * \text{К}$

Закон Стефана-Больцмана

$U = \int_0^{\inf}{U_{\lambda}(\lambda, t) d \lambda} = \sigma * t^4 * 4 / c$ - площадь под графиком распределения Планка

$I = \sigma t^4$

При росте температуры в 2 раза площадь уменьшается в 16 раз!

\subsection{Задачка}

Тело формы остывает от $T_0$ за счет лучеиспускания. Как температура зависит от времени $T(t) = ?$

$I = \sigma t^4 \dfrac{\text{Дж}}{\text{м}^2 \text{сек}}$
Обрыв связи...


\section{Фотоэффект}

$E = h\upsilon = \dfrac{hc}{\lambda}$

$I \dfrac{\text{Дж}}{\text{м}^2 \text{сек}} = Const$

Уравнение Эйнштейна

$h\upsilon - A_{\text{вых}} = \dfrac{mv^2}{2}$

$h\upsilon > A_{\text{вых}} = 3.74 \text{ЭВ}$

$i = Q / t = \dfrac{I} { h \upsilon} e$

$e \psi_{\text{запирающий}} =  \dfrac{mv^2}{2} = h\upsilon - A_{\text{вых}}$


\end{document}
