\section{Введение}

GNU Privacy Guard (GnuPG, GPG) — свободная программа для шифрования информации и создания электронных цифровых подписей. Разработана как альтернатива PGP и выпущена под свободной лицензией GNU General Public License. GnuPG полностью совместима со стандартом IETF OpenPGP. Текущие версии GnuPG могут взаимодействовать с PGP и другими OpenPGP-совместимыми системами.


\section{Работа с gpg}

Для начала требуется создать ключ с помощью команды gpg --gen-key, далее указываем размер и ждем пока ключ появится.

Далее экспортируем свой ключ в файл командой gpg --export -a > file.gpg

Получив такой ключ надо сделать gpg --import file.gpg

Найти его среди других в выводе команды gpg --list-keys

Далее подписать его своим ключом командой gpg --sign-key \$(ключ)

Экспортировать ключ тому кто прислал его на подпись командой gpg --export \$(ключ) > file.signed.gpg

Таким образом мне удалось собрать около 20 подписей под своим ключом, чего достаточно для выполнения работы.

\section{Информация о машине}

macOS Catalina 10.15.7

MacBook Pro (13-inch, 2017, Two Thunderbolt 3 ports)

Процессор 2,3 GHz 2‑ядерный процессор Intel Core i5

Память 8 ГБ 2133 MHz LPDDR3

Графика Intel Iris Plus Graphics 640 1536 МБ

\section{Заключение}

gpg позволяет отправлять надежно зашифрованные сообщения друзьям и коллегам. система подписей не позволит постороннему лицу получить данные, которые находятся в файле. Выбирая оптимальный ключ, обычно максимальной длинны свыше 4000 знаков, оценочное время дешифровки составит невероятное компьютерное время.

% \pagebreak







