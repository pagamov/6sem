%%% пример отчета взян с сайта http://k806.ystok.ru/

\documentclass[15pt]{extarticle}

\usepackage{fullpage}
\usepackage{multicol,multirow}
\usepackage{tabularx}
\usepackage{ulem}
\usepackage[utf8]{inputenc}
\usepackage[russian]{babel}
\usepackage{amsmath}
\usepackage{amssymb}

% margin from paper borders
\usepackage[margin={0.8in, 0.3in}]{geometry}
% displaystyle в каждом мас моде
\everymath{\displaystyle}

\usepackage{titlesec}

\titleformat{\section}
  {\normalfont\Large\bfseries}{\thesection.}{0.3em}{}

\titleformat{\subsection}
  {\normalfont\large\bfseries}{\thesubsection.}{0.3em}{}

\titlespacing{\section}{0pt}{*2}{*2}
\titlespacing{\subsection}{0pt}{*1}{*1}
\titlespacing{\subsubsection}{0pt}{*0}{*0}
\usepackage{listings}
\lstloadlanguages{Lisp}
\lstset{extendedchars=false,
    breaklines=true,
    breakatwhitespace=true,
    keepspaces = true,
    tabsize=2
}

\begin{document}

\section*{Отчет по лабораторной работе №\,1
по курсу \guillemotleft  Функциональное программирование\guillemotright}

\begin{flushright}
Студент группы 8О-307 МАИ \textit{Гамов Павел}, \textnumero 4 по списку \\
\makebox[7cm]{Контакты: {\tt pagamov@gmail.com} \hfill} \\
\makebox[7cm]{Работа выполнена: 25.03.2021 \hfill} \\
\ \\
Преподаватель: Иванов Дмитрий Анатольевич, доц. каф. 806 \\
\makebox[7cm]{Отчет сдан: \hfill} \\
\makebox[7cm]{Итоговая оценка: \hfill} \\
\makebox[7cm]{Подпись преподавателя: \hfill} \\

\end{flushright}

\section{Тема работы}
Примитивные функции и особые операторы Коммон Лисп

\section{Цель работы}
Цель работы: научиться вводить S-выражения в Лисп-систему, определять переменные и функции, работать с условными операторами, работать с числами, используя схему линейной и древовидной рекурсии.

\section{Задание (вариант №1.43/4)}
Даны целые числа h (0 $\leq$ h $\le$ 12) и m (0 $\leq$ m $\le$ 60), указывающие момент времени: h часов, m минут.
Запрограммируйте на языке Коммон Лисп функцию с двумя параметрами h и m, вычисляющую наименьшее время, которое должно пройти до того момента, когда часовая и минутная стрелки на циферблате совпадут. Время следует вычислять как целое число минут, округлённое в меньшую сторону.

Примеры

(parallel-hands-minutes 0 0) => 65

(parallel-hands-minutes 0 15) => 50

\section{Оборудование студента}
macOS Catalina 10.15.7 Intel Core i5 2.3 GHz 8 ГБ RAM

\section{Программное обеспечение}
macOS, среда {\tt vim + sbcl}

\section{Идея, метод, алгоритм}
Так как это мой первый опыт с данным языком, зная общие концепции я сделал прототип на Python, далее упростил его, подогнав под синтаксис. Обнаружил краевые условия, например когда оба значения равны 0, фактически они уже встретились, так что в главной функции parallel-hands-minutes я расписал два сценария, где вызываю дополнительную функцию sub-f где присутствует аккумулятор z, в котором хранится переменная для ответа.

\section{Сценарий выполнения работы}
Сделал прототип на Python. Нашел краевые условия. Создал if условие, откуда вызываю sub-f функцию. В sub-f функции оператором cond вызываю далее хвостовую рекурсивную функцию.

\section{Распечатка программы и её результаты}

\subsection{Исходный код}

\begin{lstlisting}
#!/usr/local/bin/sbcl --script
(defun sub-f (x y z)
    (cond
        ((> x 11) (sub-f 0 y z)) 
        ((> y 59) (sub-f (+ x 1) 0 z)) 
        ((and (= x (/ y 5)) (= 0 (rem y 5))) z)
        (t (sub-f x (+ y 1) (+ z 1)))))

(defun parallel-hands-minutes (x y)  
    (if (= (rem (+ y 1) 60) 0)  
        (sub-f (+ x 1) (+ y 1) 1)  
        (sub-f x (+ y 1) 1)))

(print (parallel-hands-minutes 0 0)) 
(print (parallel-hands-minutes 0 15))
\end{lstlisting}

\subsection{Результаты работы}

\begin{lstlisting}
65 
50
\end{lstlisting}

\section{Дневник отладки}

%\begin{tabular}{|c|c|c|c|}
%\hline
%Дата		&	Событие		&	Действие по исправлению	&	Примечание \\
%\hline
%\end{tabular}

\section{Замечания автора по существу работы}
sbcl предоставляет возможность отладки программы, надо лучше освоить все команды. Пока нет полного понимания языка, можно прототипировать на бумаге или другом языке.

\section{Выводы}
Я познакомился с lisp, sbcl, простыми операторами, с тем как определять и вызывать функции. Создал код использующий хвостовые рекурсии, попутно разбирая другие возможные варианты решения задачи.

\end{document}
