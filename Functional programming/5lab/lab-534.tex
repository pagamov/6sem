%%% пример отчета взян с сайта http://k806.ystok.ru/

\documentclass[15pt]{extarticle}

\usepackage{fullpage}
\usepackage{multicol,multirow}
\usepackage{tabularx}
\usepackage{ulem}
\usepackage[utf8]{inputenc}
\usepackage[russian]{babel}
\usepackage{amsmath}
\usepackage{amssymb}
\usepackage{graphicx}

% margin from paper borders
\usepackage[margin={0.8in, 0.3in}]{geometry}
% displaystyle в каждом мас моде
\everymath{\displaystyle}

\usepackage{titlesec}

\titleformat{\section}
  {\normalfont\Large\bfseries}{\thesection.}{0.3em}{}

\titleformat{\subsection}
  {\normalfont\large\bfseries}{\thesubsection.}{0.3em}{}

\titlespacing{\section}{0pt}{*2}{*2}
\titlespacing{\subsection}{0pt}{*1}{*1}
\titlespacing{\subsubsection}{0pt}{*0}{*0}
\usepackage{listings}
\lstloadlanguages{Lisp}
\lstset{extendedchars=false,
    breaklines=true,
    breakatwhitespace=true,
    keepspaces = true,
    tabsize=2
}



\begin{document}

\section*{Отчет по лабораторной работе №\,5
по курсу \guillemotleft  Функциональное программирование\guillemotright}

\begin{flushright}
Студент группы 8О-307 МАИ \textit{Гамов Павел}, \textnumero 4 по списку \\
\makebox[7cm]{Контакты: {\tt pagamov@gmail.com} \hfill} \\
\makebox[7cm]{Работа выполнена: 31.05.2021 \hfill} \\
\ \\
Преподаватель: Иванов Дмитрий Анатольевич, доц. каф. 806 \\
\makebox[7cm]{Отчет сдан: \hfill} \\
\makebox[7cm]{Итоговая оценка: \hfill} \\
\makebox[7cm]{Подпись преподавателя: \hfill} \\

\end{flushright}

\section{Тема работы}
Обобщённые функции, методы и классы объектов

\section{Цель работы}
Научиться определять простейшие классы, порождать экземпляры классов, считывать и изменять значения слотов, научиться определять обобщённые функции и методы.

\section{Задание (вариант № 5.34/2)}
Объемлющий прямоугольник для геометрической фигуры - это такой, который
полностью заключает в себе фигуру,
имеет стороны, параллельные осям координат.
Пусть класс окружность определён следующим образом:

(defclass circle ()

 ((center :initarg :center :reader center)   ; центр - экземпляр cart или polar
 
  (radius :initarg :radius :reader radius))) ; радиус - число
  
Задание: Написать обобщённую функцию и метод, возвращающий список из четырех вершин объемлющего четырёхугольника окружности.

(defgeneric containing-rect (shape))

(defmethod containing-rect ((c circle))

  ...)

\section{Оборудование студента}
macOS Catalina 10.15.7 Intel Core i5 2.3 GHz 8 ГБ RAM

\section{Программное обеспечение}
macOS, среда {\tt vim + sbcl}

\section{Идея, метод, алгоритм}
Написать два класса для разного представления точки, перегрузить функцию print, чтобы она правильно печатала тип cart.  Далее написать главную функцию обрамления окружности.

\section{Сценарий выполнения работы}
Создаем класс полярных и декартовых координат, делаем перевод из полярных в декартовы с помощью методов класса polar. Далее делаем класс circle, он имеет в себе начальную точку и радиус. Далее пишем метод обрамления круга прямоугольником. Метод простой, начиная с левой верхней точки по часовой стрелке получаем вершины прямоугольника путем сдвига на радиус центральной точки, создаем из этого массив и возвращаем.

\section{Распечатка программы и её результаты}

\subsection{Исходный код}

\begin{lstlisting}
(defclass circle ()
 ((center :initarg :center :reader center)
  (radius :initarg :radius :reader radius)))

(defgeneric containing-rect (shape))
(defmethod containing-rect ((c circle))
    (let ((x (cart-x (center c)))
          (y (cart-y (center c)))
          (r (radius c)))
        (list (make-instance 'cart :x (- x r) :y (+ y r))
              (make-instance 'cart :x (+ x r) :y (+ y r))
              (make-instance 'cart :x (+ x r) :y (- y r))
              (make-instance 'cart :x (- x r) :y (- y r)))))

(defclass polar ()
 ((radius :initarg :radius :accessor radius)
  (angle  :initarg :angle  :accessor angle)))

(defclass cart ()
 ((x :initarg :x :reader cart-x)
  (y :initarg :y :reader cart-y)))

(defmethod cart-x ((p polar))
  (* (radius p) (cos (angle p))))

(defmethod cart-y ((p polar))
  (* (radius p) (sin (angle p))))

(defmethod print-object ((c cart) stream)
  (format stream "[CART x ~d y ~d]"
          (cart-x c) (cart-y c)))

(print (containing-rect (make-instance 'circle
           :center (make-instance 'cart :x 4 :y 3)
           :radius 5)))

(print (containing-rect (make-instance 'circle
           :center (make-instance 'polar :radius 5 :angle (/ pi 4))
           :radius 5)))
\end{lstlisting}

\subsection{Результаты работы}

\begin{lstlisting}
([CART x -1 y 8] [CART x 9 y 8] [CART x 9 y -2] [CART x -1 y -2]) 

([CART x -1.4644660940672622d0 y 8.535533905932738d0]
 [CART x 8.535533905932738d0 y 8.535533905932738d0]
 [CART x 8.535533905932738d0 y -1.4644660940672627d0]
 [CART x -1.4644660940672622d0 y -1.4644660940672627d0])
\end{lstlisting}

\section{Дневник отладки}

\begin{tabular}{|c|c|c|c|}
\hline
Дата     & Событие    & Действие по исправлению   & Примечание \\
\hline
& & & \\
\hline
\end{tabular}

\section{Замечания автора по существу работы}
Замечаний не имею.

\section{Выводы}
В данной работе я познакомился с классами и методами, реализовал несколько классов, научился получать к ним доступ, изменять параметры и писать методы.

\end{document}
