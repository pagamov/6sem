% Стартовая точка:
%   x^0_1 = N (номер по списку)
%   x^0_2 = 7 (номер нашей группы)
%
% Ко всему этому можно дать некоторые скриншоты (с сайта, там графики строятся)
% Все пояснения, которые считаешь нужными - делай. Работу можешь делать как
% хочешь, можешь хоть от руки писать


\section{Метод сопряженных градиентов (2 шага)}

\subsection{Класс решаемых задач} % Какой класс задач решается этим методом

\subsection{Пример задачи} % Сформулировать свою задачу

\subsection{Что вычисляется в процессе}

\subsection{Алгоритм} % Записать четкий алгоритм

\subsection{Итерация подробно} % Описать одну итерацию подробно (первую например)

\section{Метод конфигураций (8 шага)}

\subsection{Класс решаемых задач} % Какой класс задач решается этим методом

\subsection{Пример задачи} % Сформулировать свою задачу

\subsection{Что вычисляется в процессе}

\subsection{Алгоритм} % Записать четкий алгоритм

\subsection{Итерация подробно} % Описать одну итерацию подробно (первую например)

\section{Метод Ньютона-Радсона (5 шагов)}

\subsection{Класс решаемых задач} % Какой класс задач решается этим методом

\subsection{Пример задачи} % Сформулировать свою задачу

\subsection{Что вычисляется в процессе}

\subsection{Алгоритм} % Записать четкий алгоритм

\subsection{Итерация подробно} % Описать одну итерацию подробно (первую например)

\section{Метод Нелдера-Мида (5 шагов)}

\subsection{Класс решаемых задач} % Какой класс задач решается этим методом

\subsection{Пример задачи} % Сформулировать свою задачу

\subsection{Что вычисляется в процессе}
qweqwe

\subsection{Алгоритм} % Записать четкий алгоритм

\subsection{Итерация подробно} % Описать одну итерацию подробно (первую например)
qweqwe

\section{Метод Марквардта (5 шагов)}

\subsection{Класс решаемых задач} % Какой класс задач решается этим методом
aysdas

\subsection{Пример задачи} % Сформулировать свою задачу

\subsection{Что вычисляется в процессе}
helo

\subsection{Алгоритм} % Записать четкий алгоритм

\subsection{Итерация подробно} % Описать одну итерацию подробно (первую например)
Helo
